\section{Emilia Myrta}

Math expression:

\[x^2 - y^2 = (x + y)(x - y)\]

Photo~\ref{fig:dog}.

\begin{figure}[htbp]
    \centering
    \includegraphics[width=0.4\textwidth]{Pictures/zdjecie_emilia.jpg}
    \caption{Dog.}
    \label{fig:dog}
\end{figure}

Table~\ref{tab:test}.

\begin{table}[htbp]
\centering
\begin{tabular}{|l|l|l|l|l|l|}
\hline
Column 1 & Column 2 & Column 3 & Column 4 & Column 5 & Column 6 \\ \hline
Row 2    & 0        & 1        & 2        & 3        & 4        \\ \hline
Row 3    & 1        & 2        & 3        & 4        & 5        \\ \hline
\end{tabular}
\label{tab:test}
\caption{}
\end{table}

List of largest cities by population (2018):
\begin{enumerate}
  \item Tokyo, Japan - 37,468,000 
  \item Delhi, India - 28,514,000 
  \item Shanghai, China	- 25,582,000 	
\end{enumerate}

\begin{itemize}
  \item Number 1
  \item Number 2
  \item Number 3
\end{itemize}

\begin{flushleft}
Boosting and encouraging use of \underline{energy efficiency technologies} would lead to reduced energy needs for powering, heating, and cooling of homes, businesses, and industries. This would be effective in \textbf{reducing global warming} as the problem is largely contributed to by the energy used for cooling, heating, and power services in industries, businesses, and homes. In the transportation sector for instance, \underline{switching to fuels that are low in carbon}, and improving fuel efficiency in terms of miles per gallon would reduce the amount of heat-trapping emissions released into the atmosphere.\par
Additionally, \textbf{\textit{revving up renewable energy}} could reduce global warming. The vast majority of energy needs worldwide can be potentially met by such renewable sources of energy as bioenergy, geothermal, wind, and solar energy that apart from reducing pollution, would also \emph{create jobs}. According to the Environmental Protection Agency’s 2012 report, \textbf{coal-fired power plants produce approximately \emph{25 percent} of total U.S. global warming emissions} while natural gas-fired power plants produce 6 percent of total emissions. In contrast, most renewable energy sources produce little to no global warming emissions. Conclusively, boosting energy efficiency and adopting renewable energy would reduce global warming.
\end{flushleft}
