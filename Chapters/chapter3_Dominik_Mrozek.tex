\newpage
\section{Dominik Mrozek}

Fundamental trigonometric equation:
\[(sinx)^2+(cosx)^2=1\]

Photo~\ref{fig:panda}.
\begin{figure}[htbp]
    \centering
    \includegraphics[width=1\textwidth]{Pictures/panda.jpg}
    \caption{Panda eating bamboo.}
    \label{fig:panda}
\end{figure}

Table~\ref{tab:tabelka}.
\begin{table}[htbp]
\centering
\begin{tabular}{|l|l|l|}
\hline
\textbf{\#} & \textit{\textbf{Country}} & \textit{\textbf{GDP per Capita (US \$)}} \\ \hline
\textbf{1}  & Monaco                    & 190,512                                  \\ \hline
\textbf{2}  & Liechtenstein             & 180,366                                  \\ \hline
\textbf{3}  & Luxembourg                & 115,873                                  \\ \hline
\textbf{4}  & Switzerland               & 87,097                                   \\ \hline
\textbf{5}  & Macao                     & 86,117                                   \\ \hline
\textbf{6}  & Ireland                   & 85,267                                   \\ \hline
\textbf{7}  & Norway                    & 67,389                                   \\ \hline
\textbf{8}  & United States             & 63,543                                   \\ \hline
\textbf{9}  & Denmark                   & 61,063                                   \\ \hline
\textbf{10} & Singapore                 & 59,797                                   \\ \hline
\end{tabular}
\caption{Table of top 10 countries with greatest GDP per Capita}
\label{tab:tabelka}
\end{table}

\newpage
List of 10 biggest countries:
\begin{enumerate}
  \item Russia 
  \item Canada 
  \item China 	
  \item United States 
  \item Brazil
  \item Australia 
  \item India
  \item Argentina 
  \item Kazakhstan
  \item Algeria
\end{enumerate}

Popular internet browsers:
\begin{itemize}
  \item Google Chrome
  \item Safari
  \item Edge
  \item Firefox
  \item Internet Explorer
  \item Opera
\end{itemize}

\def\ind{\quad}
\textbf{Mathematics} (from Ancient Greek máthēma: \underline{'knowledge, study, learning'}) is an area of knowledge 
that includes such topics as numbers (\textbf{arithmetic} and \textbf{number theory}), formulas and related structures (\textbf{algebra}), shapes and the spaces in which they are contained (\textbf{geometry}) and quantities and their changes (\textbf{calculus and analysis}).
\endgraf
\def\ind{\quad}
Most mathematical activity involves the \textbf{discovery of properties of abstract objects and the use of pure reason to prove them}. These objects consist of either abstractions from nature or—in modern mathematics—entities that are stipulated with certain properties, called \textbf{axioms}. A proof consists of a succession of applications of deductive rules to already established results. These results include previously proved theorems, axioms, and—in case of abstraction from nature—some basic properties that are considered as true starting points of the theory under consideration.
